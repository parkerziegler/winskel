\documentclass{article}

\usepackage{amsmath}
\usepackage{amssymb}

\begin{document}
  \newcommand{\N}{\mathbb{N}}

  \noindent \textbf{Exercise 3.1.} Prove by mathematical induction that the following property $P$ holds for all natural numbers.
  $$
  P(n) \iff \Sigma^n_{i = 1} (2i - 1) = n^2
  $$
  \noindent (The notation $\Sigma^l_{i = k} s_i$ abbreviates $s_k + s_{k + 1} + s_{k + 2} + ... + s_l$ when $k, l$ are integers with $k < l$).

  \bigskip
  \noindent We proceed by mathematical induction on $n$.

  \bigskip
  \noindent \textbf{Base Case}

  \medskip
  \noindent We choose $P(1)$ as the base case. $P(0)$ is vacuously true; $i$ begins at 1, so $\Sigma^0_{i = 1}(2i - 1)$ is not defined.

  \begin{align*}
    P(1) &\iff \Sigma^1_{i = 1}(2i - 1) = 1^2 \\
    &\iff (2(1) - 1) = 1^2 \\
    &\iff 1 = 1
  \end{align*}

  \bigskip
  \noindent \textbf{Inductive Case}

  \medskip
  \noindent Assume the proposition $P$ holds for an arbitrary natural number $m$. We need to show that the proposition $P(m + 1)$ also holds.

  \begin{align*}
    P(m + 1) &\iff \Sigma^{m + 1}_{i = 1}(2i - 1) = (m + 1)^2 \\
    &\iff \Sigma^{m}_{i = 1}(2i - 1) + (2(m + 1) - 1) = (m + 1)^2 \\
    &\iff m^2 + 2m + 2 - 1 = (m + 1)^2 \\
    &\iff m^2 + 2m + 1 = m^2 + 2m + 1 \\
  \end{align*}

  \noindent Thus: $\forall m \in \N, P(m) \implies P(m + 1)$. We've proven the base case and the inductive case, completing the proof. QED

  \bigskip
  \noindent \textbf{Exercise 3.2} A string is a sequence of symbols. A string $a_1a_2...a_n$ with $n$ positions occupied by symbols is said to have length $n$. A string can be empty in which case it is said to have length 0. Two strings $s$ and $t$ can be concatenated to form the string $st$. Use mathematical induction to show there is no string $u$ which satisfies $au = ub$ for two distinct symbols $a$ and $b$.

  \bigskip
  \noindent We proceed by mathematical induction on the length of $u$.

  \bigskip
  \noindent \textbf{Base Case}

  \medskip
  \noindent For the base case, we choose $u$ to be the empty string, which has length 0 by definition. Therefore:
  
  $$
  au = a, ub = b
  $$
  
  \medskip
  \noindent Since $a$ and $b$ are two distinct strings by definition ($a \neq b$), we can apply the transitive property to derive the following:

  $$
  au = a \neq b = ub \implies au \neq ub
  $$

  \bigskip
  \noindent \textbf{Inductive Case}

  \medskip
  \noindent For the inductive case, we need to show that, given an arbitrary string $u$ of arbitrary length $n$ satisfying $au \neq ub$, $au' \neq u'b$ holds with an arbitrary string $u'$ of length $n + 1$. This is the inductive hypothesis.

  \medskip
  \noindent Assume we make a single character addition to $u$ to produce $u'$ with length $n + 1$. Then we have that $au \neq au'$ and $ub \neq u'b$, because $u$ and $u'$ have different lengths by definition and therefore cannot be concatenated to produce the original strings $au$ and $ub$. Applying our inductive hypothesis and the transitive property, we can then derive the following:

  $$
  au' \neq au \neq ub \neq u'b \implies au' \neq u'b
  $$

  \medskip
  \noindent We've proven the base case and the inductive case, completing the proof. QED
  
  \bigskip
  \noindent \textbf{Exercise 3.4} Prove by structural induction that the evaluation of arithmetic expressions always terminates, i.e., for all arithmetic expression $a$ and states $\sigma$ there is some $m$ such that $\langle a, \sigma \rangle \Downarrow m$.

  \medskip
  \noindent We proceed by structural induction on cases of arithmetic expressions \textbf{AExp}.

  \medskip
  \noindent \textbf{Numeric literals $m$}. If $a$ is a numeric literal $m$, the only rule for its evaluation is $\langle a, \sigma \rangle \Downarrow m$. Thus, we've shown that $\langle a, \sigma \rangle \Downarrow m$ exists for this case.

  \medskip
  \noindent \textbf{Addition $a_0 + a_1$}. If $a$ is an addition expression, then its evaluation matches the structure of the following rule:

  $$
  \frac{\langle a_0, \sigma \rangle \Downarrow m_0 \hspace{5mm} \langle a_1, \sigma \rangle \Downarrow m_1}{\langle a_0 + a_1, \sigma \rangle \Downarrow m}
  $$

  \noindent By the induction hypothesis applied to $a_0$ and $a_1$ we obtain the numbers $m_0$ and $m_1$, which together sum to the number $m$.

  \medskip
  \noindent \textbf{Subtraction $a_0 - a_1$}. If $a$ is a subtraction expression, then its evaluation matches the structure of the following rule:

  $$
  \frac{\langle a_0, \sigma \rangle \Downarrow m_0 \hspace{5mm} \langle a_1, \sigma \rangle \Downarrow m_1}{\langle a_0 - a_1, \sigma \rangle \Downarrow m}
  $$

  \noindent By the induction hypothesis applied to $a_0$ and $a_1$ we obtain the numbers $m_0$ and $m_1$, whose difference is the number $m$.

  \medskip
  \noindent \textbf{Multiplication $a_0 \times a_1$}. If $a$ is a multiplication expression, then its evaluation matches the structure of the following rule:

  $$
  \frac{\langle a_0, \sigma \rangle \Downarrow m_0 \hspace{5mm} \langle a_1, \sigma \rangle \Downarrow m_1}{\langle a_0 \times a_1, \sigma \rangle \Downarrow m}
  $$

  \noindent By the induction hypothesis applied to $a_0$ and $a_1$ we obtain the numbers $m_0$ and $m_1$, whose product is the number $m$.
  
  This completes all cases $a \in \textbf{AExp}$ and proves that:

  $$
  \forall a \in A. \exists m~|~\langle a, \sigma \rangle \Downarrow m
  $$
\end{document}